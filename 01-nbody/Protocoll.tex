\documentclass[ngerman]{article}
\usepackage[utf8]{inputenc}
\usepackage{babel}
\usepackage[T1]{fontenc}
\usepackage{lmodern}
\usepackage{amssymb}
%\usepackage{xcolor}
\usepackage{amsmath}
\usepackage{physics}
\newcommand{\equalhat}{\overset{\scriptscriptstyle\wedge}{=}}
\usepackage{graphicx}
\usepackage[a4paper,top=2cm,bottom=2cm,left=2cm,right=2cm,marginparwidth=1.75cm]{geometry}
\usepackage[hidelinks]{hyperref}
\usepackage{multirow}
\usepackage{siunitx}
\sisetup{locale = DE}
\usepackage{multirow}
\usepackage{float}
\usepackage{fancyhdr}
\usepackage{subcaption}
\usepackage[version=4]{mhchem}

\title{\textbf{The Classical N-Body-Problem }\\  
Eberhard Karls Universität Tübingen- Physikalisches Institut\\ Computational Physics
}
%\begin{center}
\author{ Noah Matera, Richard Abele \\ 
Version 1
\\  } 

\date{\item{Session Date Tübingen, 21.10.24}\\ \item{Hand-in Tübingen, den} \\ Wintersemester 24/25}
%\end{center}
\begin{document}
\thispagestyle{empty}
\maketitle
\newpage

\pagestyle{fancy}                    % Eigener Seitenstil
\fancyhf{}                           % Alle Kopf- und Fußzeilenfelder bereinigen
\fancyhead[L]{Abele,Matera}                % Kopfzeile links
\fancyhead[C]{Integrators}                        % Zentrierte Kopfzeile
\fancyhead[R]{21.10.24}                 % Kopfzeile rechts
\renewcommand{\headrulewidth}{0.4pt} % Obere Trennlinie
\fancyfoot[C]{\thepage}              % Seitennummer
\renewcommand{\footrulewidth}{0pt}
\tableofcontents 
\newpage

\section{Background}
The following exercises showcase several integrators to solve the n-body problem. 
For simplicity, the total mass $M$ as well as the gravitational constant $G$ are set to 1 and the coordinates are shifted to a center of mass coordinate system, to counteract any drifting of the center of mass.
We were provided with initial conditions, that is, mass, velocity, and location, for $n=2, 3, 100, 1000$ 3-dimensional particles.

\section{Exercise 1: The Program}
Our program is able to normalize the data, meaning transforming it into a center of mass coordinate system. Then those initial conditions can be used to simulate the time evolution using the following integrators: Euler, Euler-Cromer, Velocity-Verlet, 
Heun, RK4, Hermite, and iterated Hermite.

\textbf{explain code here and maybe put github link and explanation of how it works here}


\subsection{Exercise 2: The 2-Body-Problem}  
In order to analyze the effectiveness of our integrators on the 2-Body-Problem, we analyze the time evolution of the energy $E$, angular momentum $j$, Runge-Lenz vector $e$, and the major axis of the orbit $a_e$. These are all plotted logarithmically against our pseudo time for each integrator.

%\begin{figure}[H]
%    \centering
%    \includegraphics[width=0.5\linewidth]{title.png}
%    \caption{Plot}
%    \label{Plot1}
%\end{figure}

\subsection{Exercise 3: The N-Body-Problem}  
For $n=100, 1000$, just the energy is plotted logarithmically as
\begin{align}
    \log(\frac{E_{\text{initial}}-E}{E_{\text{initial}}}).
\end{align}





We also compared the runtime difference for $n=100$ and $n=1000$  for 2 exemplary integrators
\section{Fazit}
\section{Quellenverzeichnis} 
\begin{enumerate}
    \item  Versuchsanleitung des Physikalischen Institutes der Universität Tübingen:  Projekt 1: Das klassische N-Körper-Problem \label{Versuchsanleitung}
\end{enumerate}
\end{document}
